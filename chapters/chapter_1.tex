\chapter{مقدمه} 
\newpage


\section{کلیات}
\noindent {
شبکه‌های اجتماعی جزو جدا نشدنی زندگی جامعه انسانی می باشند. در گذشته این شبکه‌ها با محدودیت‌های زیادی جهت مطالعه روبرو بوده اند. با گسترش روزافزون شبکه‌های اجتماعی آنلاین در ده سال گذشته بسیاری از محدودیتهای ارتباطی ناشی از فواصل زیاد جغرافیایی اعضا ازبین رفته است، که باعث بزرگ شدن اندازه ی شبکه‌های اجتماعی مورد بحث شده است \cite{musial_social_2013}. همین طور تکنولوژی لازم برای ذخیره و بازیابی و واکاویی اطلاعات مربوط به این شبکه‌های عظیم و کاربرانشان در دسترس می‌باشد.
\\
\indent
نمود شبکه‌های اجتماعی مورد بحث سایت‌هایی مانند فیسبوک و گوگل‌پلاس و اورکوت می‌باشند. این سایت‌هابه کاربران خود سرویس هایی ارایه می دهند که به کمک این سرویس ها اعضا قادر به اشتراک گذاری محتوای اجتماعی خودشان با دیگران می‌باشند. 
محتوای این افراد در ظرف هایی مانند متن و تصویر و ویدیو در سطح این شبکه‌ها پخش می شود. این امر از این جهت برای محققان مورد علاقه است که محتوای اجتماعی در جریان در این سایت‌ها قدرت تاثیرگذاری روی رفتار و منش اعضای این شبکه‌ها را به سادگی داراست. از طرف دیگر محتوای مورد توجه یک کاربر می‌تواند علایق و تمایلات وی به خوبی نشان دهد، از این رو تحقیق و پژوهش در امر چگونگی فرایند پخش محتوای اطلاعاتی در سطح شبکه های اجتماعی آنلاین مورد توجه محققانی از رشته‌های ‌مختلف مانند جامه‌شناسی و اقتصاد و کامپیوتر بوده است.
\\
\indent
اولین کارهای مربوط به امر پخش اطلاعات به اوایل قرن بیستم در باره‌‌ی چگونگی روند بکارگیری فناوری های نوین بوده است \cite{rogers2010diffusion}. برای نمونه به چگونگی پخش امر استفاده‌ی کشاورزان از دانه‌های اصلاح شده‌ی محصولات به جای دانه‌های مرسوم در چند ایالت آمریکا در دهه‌ی ۳۰ قرن گذشته.


}

\section{هدف سمینار}

در این سمینار به بررسی اجزای مهم تشکیل دهنده‌ی فرایند پخش اطلاعات می پردازیم. هم‌چنین مدل‌های ارایه شده برای مدل‌سازی و تحلیل این فرایند را مورد بررسی قرار خواهیم داد. البته پخش اطلاعات را می‌توان از زوایای مختلفی مورد مطالعه و تحقیق قرار داد، مانند طراحی و ساخت مدل‌های توصیفی از فرایند پخش اطلاعات، بررسی موضوعات متنی مورد توجه\cite{guille_information_2013}، یافتن افراد تاثیرگذار \cite{cha_measuring_2010} و پیش‌بینی انتشار\cite{cheng_can_2014} که در این نوشته بیشتر به بعد مدل‌سازی این فرایند پرداخته شده است.
\section{ساختار این گزارش}
\begin{persian}
\noindent
فصل دوم این نوشته به آشنایی با مفاهیم و مطالب مرتبط پیرامون پخش شدن اطلاعات در سطح شبکه‌های اجتماعی آنلاین می‌پردازد تا خواننده را با کلیت موضوع مورد‌ بحث و موارد مطرح در زمینه‌ی پژوهش ‌و تجزیه‌ و تحلیل فرایند‌ عنوان این نوشته به طور عام آشناتر کند، هم‌چنین به معرفی چند جز تاثیرگذار امر پخش‌ اطلاعات در سطح شبکه‌های اجتماعی آنلاین خواهیم پرداخت. لذا در صورت آشنایی با موارد مطرح در این فصل خواننده می‌تواند از این مطالب بگذرد و به فصل دوم برای ادامه‌ی مطالعه مراجعه کند.
\\
\indent 
در فصل سوم نیز به معرفی و مقایسه‌ی مدل‌های مطرح ‌شده برای مدل‌سازی چگونگی فرایند پخش و همه‌گیری\پانویس{\lr{ Epidemic}} و شایعه به طور عام و طور خاص‌تر کاربری این مدل‌ها در مدل‌سازی فرایند پخش اطلاعات در سطح شبکه‌های اجتماعی آنلاین اختصاص داده شده است. فصل چهار هم به جمع‌بندی و نتیجه‌گیری مطالب گفته‌شده در فصول پیشین خواهد پرداخت.
 
\end{persian}

