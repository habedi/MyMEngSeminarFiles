\chapter{نتیجه گیری} 

\newpage

\section{جمع بندی}
\noindent{
شبکه‌های اجتماعی خصوصیات منحصر به فردی دارند که آن‌ها را از دیگر شبکه‌ها متمایز می‌سازد. این خوصیات در بسیاری از لحاظ برای تولید مدل‌های پخش اطلاعات ساده سازی شده اند. به طور کلی از میان روش های زیادی که برای مدل‌کردن فرایند پخش اطلاعات در شبکه‌های اجتماعی مطرح شده اند دو مدل \lr{LT} و \lr{IC} مورد استقبال زیادی قرار گرفته اند. در این دو مدل که برای بار اول در سال ۲۰۰۳ برای ارایه مدلی برای حداکثر نمودن تاثیر اجتماعی میان اعضای یک شبکه‌ی اجتماعی ارایه گردیده اند، برای یافتن $k$ فردی که در پایان شبیه سازی تاثیر را به حداکثر برسانند الگوریتم‌های هیوریستیک زیادی مطرح شده اند که در حدود 63 درصد را پوشش می دهند. لذا اکثر این الگوریتم‌ها از مشکل مقیاس‌ناپذیری\پانویس{ \lr{Non-Scalablility}} در بعد اندازه‌ گراف مورد مطالعه رنج می برند. \\
در کنار این دو مدل‌های ریاضی همه‌گیری بیماری که معروف ترینشان مدل‌های \lr{SIR} و \lr{SIRS} و \lr{SIS} و \lr{SI} می‌باشند در چند سال اخیر برای مدل‌کردن فرایند پخش اطلاعات از مدل کردن پخش شایعه تا مدل کردن انتشار رفتار و خبر استفاده شده اند. این محققان در این مدل‌ها و مشتقات آن‌ها تغییراتی وارد نموده اند تا در بستر گراف یک شبکه‌ی اجتماعی قابل استفاده باشند. 
 
}
\section{کار های آتی}
\noindent{
اکثر مدل‌های پخش اطلاعات که در این گزارش ذکر شدند به صورت موضعی و بر روی مجموعه‌های مشخصی از داده‌های شبکه‌هایی هم چون توییتر و فیسبوک و چند شبکه اجتماعی معروف دیگر ازمایش گردیده اند. لذا تعمیم این که رفتار کاربران همه‌ی شبکه‌های اجتماعی موجود در اینترنت با وجود امکانت متنوع و گستردگی زیاد سرویس‌های آن‌ها در فرایند پخش اطلاعات به یک گونه باشد کار بیهوده‌ای است. از طرفی خیلی از این مدل‌ها یک تصویر لحظه‌ای از شبکه اجتماعی را مورد مطالعه قرار می‌دهند، ضمنا این مدل‌ها بسیاری از خصوصیات اصلی یک شبکه اجتماعی مثل هوموفیلی میان کاربران و قدرت اتصال‌های میان افراد را در کار خود تاثیرگذار نمی‌کنند. برای همین این مدل‌ها خوصیات منحصر به فرد شبکه ‌های اجتماعی که‌ آن‌ها را از مابقی شبکه‌های ارتباطی و شبکه‌های دنیای واقعی متمایز می‌سازند در مدل خود به خوبی لحاظ ننموده اند.
\\
از طرفی بحث این‌که آیا امکان پیش‌بینی امر انتشار چیزی مانند شایعه در یک شبکه‌ی اجتماعی وجود دارد موردیست که در چند سال اخیر بسیار مورد توجه قرارگرفته است ولی کار انجام شده از لحاظ طراحی مدل‌های بهتر و دقیق تر با فرض قابل پیش‌بینی بودن امر انتشار امکان بهبود را دارد. ضمنا موضوعاتی مانند یافتن افراد تاثیر گذار در امر پخش اطلاعات، همین طور ارایه مدل‌هایی که پویایی گراف شبکه‌ی اجتماعی را به خوبی دربر بگیرند نیز می‌توانند از موضوعات جذاب برای پژوهش در این زمینه باشند.
}

